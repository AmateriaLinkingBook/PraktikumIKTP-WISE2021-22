\section{Einführung}
Die Weitwinkel-Compton-Koinzidenz-Methode (WCKM) ist eine Methode zur Energiekalibrierung von vor allem organischen Szintillatoren. Da organische Szintillatoren Atome mit niedrigen Kernladungszahlen (Niedrig-Z-Szintillatoren) verwenden kann keine Kalibrierung mittels Bestimmung des Vollenergiepeaks stattfinden, denn für die typischerweise verwendeten Kalibrierenergien (\SI{0.5}{\mega\electronvolt} bis \SI{1.5}{\mega\electronvolt}) überwiegt bei niedrigem Z die Compton-Streuung gegenüber dem Photoeffekt.

Die zu bearbeitenden Aufgaben sind:
\begin{enumerate}
    \item Vergleich der Detektorspektren
    \item Energiekalibrierung des HPGe-Detektors
    \item Untersuchung einzelner Streuwinkel
    \item Energiekalibrierung des organischen Szintillators
\end{enumerate}
