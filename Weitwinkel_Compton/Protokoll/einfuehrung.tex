\section{Einführung}
In diesem Praktikumsversuch soll ein Photonendetektor mittels Weitwinkel-Compton-Koinzidenz-Methode (WCKM) kalibriert werden.
Die WCKM ist eine Methode zur Energiekalibrierung von vor allem organischen Szintillatoren. Da organische Szintillatoren Atome mit niedrigen Kernladungszahlen (Niedrig-Z-Szintillatoren) verwenden kann keine Kalibrierung mittels Bestimmung des Vollenergiepeaks stattfinden, denn für die typischerweise verwendeten Kalibrierenergien (\SI{0.5}{\mega\electronvolt} bis \SI{1.5}{\mega\electronvolt}) überwiegt bei niedrigem Z die Compton-Streuung gegenüber dem Photoeffekt.
Für den Versuch wird außerdem ein HPGe-Detektor kalibriert um den Detektor auf Basis des organischen Szintillators mit diesem zu vergleichen. Der HPGe-Detektor kann aufgrund seiner höheren Kernladungszahl mittels Vollenergiepeakanalyse kalibriert werden.
Der Versuch wird mithilfe eines Roboterarms als Probenhalter ausgeführt was das anfahren von präzisen Koordinaten ermöglicht und zusätzlichen Strahlenschutz gewährleistet.

Die zu bearbeitenden Aufgaben sind konkret:
\begin{enumerate}
    \item Vergleich der Detektorspektren
    \item Energiekalibrierung des HPGe-Detektors
    \item Untersuchung einzelner Streuwinkel
    \item Energiekalibrierung des organischen Szintillators
\end{enumerate}
