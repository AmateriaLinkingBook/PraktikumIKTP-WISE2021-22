\subsection{Untersuchung einzelner Streuwinkel}

Zur Kalibrierung des organischen Szintillators wurde die WCKM verwendet. Dafür wurden immer koinzidente Ereignisse aufgenommen und die gemessenen Detektorkanäle in einem zweidimensionalen Histogramm gegeneinander Aufgetragen. Vor der eigentlichen Kalibrierung wurden jedoch einzelne Streuwinkel untersucht. Dafür wurde die $^{22}$Na Probe mithilfe des Roboterarms an verschiedene Koordinaten gefahren (s. Tabelle \ref{kali_szint_coords}). Die entsprechenden Histogramme sind in Abbildung \ref{kali_szint_winkelabh} zu sehen.

\begin{table}[h]
    \centering
    \begin{tabular}{|c | c |c |}
        \hline
        Koordinaten $(x,y,z)$  & Streuwinkel & Messzeit \\
        \hline
        $(-33, 10, 7)$ & \SI{70.7}{\degree} & \SI{60}{\minute} \\
        $(-33, 4, 15)$ & \SI{60.0}{\degree} & \SI{30}{\minute} \\
        $(-33, 6, 20)$ & \SI{50.2}{\degree} & \SI{30}{\minute} \\
        $(-33, 9, 24)$ & \SI{41.2}{\degree} & \SI{30}{\minute} \\
        $(-33, 20, 27)$ & \SI{20.3}{\degree} & \SI{30}{\minute} \\
        \hline
    \end{tabular}
    \caption{Angefahrene Koordinaten der $^{22}$Na Probe, daraus resultierender Streuwinkel (grob) und Messzeit an dem jeweiligen Punkt}
    \label{kali_szint_coords}
\end{table}

\begin{figure}[ht]
	\centering
    \begin{tabular}{c c}
        \includegraphics[width=0.49\textwidth]{images/kali_szint_Streuwinkel_70.png} & \includegraphics[width=0.49\textwidth]{images/kali_szint_Streuwinkel_60.png} \\ \includegraphics[width=0.49\textwidth]{images/kali_szint_Streuwinkel_50.png} & \includegraphics[width=0.49\textwidth]{images/kali_szint_Streuwinkel_40.png} \\\
        \includegraphics[width=0.49\textwidth]{images/kali_szint_Streuwinkel_20.png} & \\
    \end{tabular}
	\caption{Histogramme der koinzidenten Ereignisse für verschiedene Streuwinkel im Szintillator bei nutzung einer $^{22}$Na Probe, für den Streuwinkel von \SI{50.2}{\degree} ist die gesuchte lineare Beziehung der Detektorkanäle bei koinzidenten Ereignissen eingetragen}
	\label{kali_szint_winkelabh}
\end{figure}

Man erkennt in diesen Histogrammen einige Dinge. Zum ersten verschiebt sich mit sinkendem Streuwinkel der Bereich in dem besonders viele Ereignisse auftreten zu höheren Kanälen im HPGe-Detektor und zu niedrigeren im Szintillationsdetektor. Auch sieht man sofort den linearen Zusammenhang der gemessenen Kanäle (Energien). Diesen kann man beobachten für jede von der Strahlungsquelle abgegebene Energie der Photonen, jedoch ist die im Bild rechte Gerade deutlich schwerer zu erkennen. Man kan daraus folgern das wie bereits diskutiert bei koinzidenten Ereignissen häufig die Energie die bei einer Compton-Streuung nicht im Szintillator deponiert wird im HPGe-Detektor deponiert wird. Man kann anhand der Intensitäten der Punkte auf der Geraden (siehe Farbskala) außerdem darauf schließen, dass ein Streuen in kleineren Winkeln wahrscheinlicher ist und damit in der selben Messzeit öfter auftritt. Man muss dazu erwähnen das für den Streuwinkel von \SI{70.7}{\degree} eine Bleiverschirmung zwischen Quelle und Szintillator stand womit dieses Histogramm für diese Aussage nicht herangezogen werden kann.

\subsection{Energiekalibrierung des organischen Szintilators}

Für die Kalibrierung wurden die diagonalen, linearen Bereiche gefittet und dann der Schnittpunkt der Geraden mit der y-Achse gesucht. Dies entspräche dann nämlich dem Fall das die im HPGe-Detektor deponierte Energie $0$ ist und sämtliche Energie des Photons im Szintillator deponiert wurde. Ein Beispiel für einen Fit ist in Abb. \ref{kali_szint_bsp_fit} zu sehen.

\begin{figure}[ht]
	\centering
    \includegraphics[width=0.98\textwidth]{images/kali_szint_fit_Na22.png}
	\caption{Beispiel für die verwendete Prozedur der linearen Regression für Nutzung einer $^{22}$Na Probe. die Zuerst wurde der grobe Bereich des linearen Zusammenhangs ausgeschnitten. Danach wurde eine lineare Regression der Daten durchgeführt. Zuletzt wurden zwei Geraden äquidistant und parallel zur Regressionsgeraden verschoben bis $ 68 \% $ ($1 \sigma$) aller Punkte des ausgeschnittenen Bereichs innerhalb dieser Geraden liegen. Die Schnittpunkte dieser Geraden mit der y-Achse stellen unsere Messunsicherheit dar.}
	\label{kali_szint_bsp_fit}
\end{figure}

Bei der Kalibrierung mit $^{22}$Na wurden die Daten der Streuwinkelmessung und eine zweistündige Messung bei der die Probe in einem Kreisbogen gefahren wurde benutzt. Bei der Kalibrierung mit $^{133}$Ba wurde die Probe \SI{4.5}{\hour} im Kreisbogen um den Szintillator gefahren. Die Ergebnisse des Fittings sind in Tabelle \ref{kali_szint_Energien} zu sehen.

\begin{table}[h]
    \centering
    \begin{tabular}{|c | c | c | c|}
        \hline
        Element & Photonenenergie & Kanal & Unsicherheit im Kanal \\
        \hline
        $^{22}$Na & \SI{511.00}{\kilo\electronvolt} & 346 & 14 \\
        $^{22}$Na & \SI{1274.54}{\kilo\electronvolt} & 906 & 15 \\
        $^{133}$Ba & \SI{356.02}{\kilo\electronvolt} & 229 & 9 \\
        \hline
    \end{tabular}
    \caption{Zerfallsenergien und zugeordnete Kanäle im Szintillationsdetektor. Leider ist nur eine der Linien der $^{133}$Ba Quelle in den Daten gut zu erkennen.}
    \label{kali_szint_Energien}
\end{table}

Auch hier wird wieder ein linearer Zusammenhang zwischen Kanälen und Energien angenommen. Dieser Fit wurde mithilfe des in \cite{Fit_bivariate} beschriebenen Algorithmus nach York durchgeführt, wobei für die Fehler in der Energie extrem kleine Werte angenommen wurden (da die Photonenenergien sehr gut bekannt sind), sodass diese keinen Einfluss auf das Ergebnis haben.
Damit ergibt sich die Kalibrierung für den Szintillator:

\begin{gather}
    E_{\text{Szintillator}} (K) = [(1.357 \pm 0.035) \cdot K + (44.199 \pm 16.521)] \si{\kilo\electronvolt}
\end{gather}

Wobei natürlich $K$ der Kanal des ADC ist.
Der relative Fehler des Anstiegs ist also etwa \SI{3}{\percent} und der der Verschiebung etwa \SI{37}{\percent}. Damit ist gerade letztere recht ungenau. Dies liegt zum größten Teil an der im Histogramm erkennbaren Breite der diagonalen Geraden. Die zufälligen Koinzidenzen spielen hier eine untergeordnete Rolle. Woher diese Breite der Diagonalen kommt ist mit den bisherigen Betrachtungen nicht zu sagen. Der Ursprung könnte in der Elektronik liegen, z.B. im Photoelektronenvervielfacher.
