\section{Diskussion}

Mithilfe einer Vollenergiepeak-Analyse wurde die Kalibriergerade des HPGE-Detektors bestimmt zu:
\begin{gather}
    E_{\text{HPGe-Detektor}} (K) = [(0.925 \pm 0.028) \cdot K + (-1.155 \pm 8.678)] \ \si{\kilo\electronvolt}
\end{gather}
Zusätzlich wurde die Kalibriergerade des organischen Szintilators mithilfe der Weitwinkel-Compton-Methode bestimmt zu:
\begin{gather}
    E_{\text{Szintillator}} (K) = [(1.357 \pm 0.035) \cdot K + (44.199 \pm 16.521)] \ \si{\kilo\electronvolt}
\end{gather}
Die Gerade des HPGE-Detektors weist nur eine geringe Unsicherheit auf, was im Anbetracht des etablierten Verfahrens zu erwarten war.
Eine weitere Verbesserung der Genauigkeit, wäre eventuell mit Berücksichtigung der Untergrundstrahlung zu erreichen oder durch Verwendung weiterer Strahlenquellen zur Kalibrierung.

Auch die Kalibrierung des organischen Szintilators hat keine allzu große Unsicherheit.
Zur Verbesserung wäre es naheliegend, die Koinzidenzspektren zu verbessern.
Zum Beispiel könnten stärkere Strahlenquellen genutzt werden, oder man könnte, was allein schon aus Gründen des Strahlenschutzes zu bevorzugen wäre, längere Messzeiten ansetzen.
Auch hier sollte sich eine Beseitigung der Untergrundstrahlung oder das verwenden weiterer Quellen positiv auswirken.
Speziell die Nutzung weiterer Quellen mit Strahlung im höheren Energiebereich sollte eine Verbesserung der Genauigkeit dort erreichen.

Weiterhin wäre auch eine Berücksichtigung der Linienbreite einen Gedanken wert.
Dabei sollte es auch möglich sein, die Energieauflösung des Szintilators anhand der Verteilung entlang der Geraden zu analysieren, was wiederum für die Energiekalibrierung genutzt werden könnte.

Letztendlich ist die Genauigkeit der verwendeten Methode auch aus rein technischen Gründen beschränkt.
Es wäre auch möglich andere Methoden zu testen, z. B. indem ausgenutzt wird, dass sich die Energien des HPGE-Detektors und des Szintilators entlang der Geraden zur ursprünglichen Photonenenergie addieren.
Dabei würde dann eine lineare Regression innerhalb der Koinzidenzspektren entfallen und als Alternative die Punkte entlang der Geraden direkt genutzt werden.

Insgesamt ist der Versuch gelungen.
Offenbar ist eine Kalibrierung von organischen Szintillatoren mithilfe der WCKM mit guter Genauigkeit möglich.
