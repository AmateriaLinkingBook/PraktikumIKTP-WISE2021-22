% Vorlage: https://www.pfsr.de/latex

% -- Anfang Präambel
\documentclass[german,  % Standardmäßig deutsche Eigenarten, englisch -> english
parskip=full,  % Absätze durch Leerzeile trennen
%bibliography=totoc,  % Literatur im Inhaltsverzeichnis (ist unüblich)
%draft,  % TODO: Entwurfsmodus -> entfernen für endgültige Version
]{scrartcl}

\usepackage[utf8]{inputenc}  % Kodierung der Datei
\usepackage[T1]{fontenc}  % Vollen Umfang der Schriftzeichen
\usepackage[main=english, ngerman]{babel} % Sprache auf Deutsch (neue Rechtschreibung)

% Mathematik und Größen
\usepackage{amsmath}
\usepackage[locale=DE,  % deutsche Eigenarten, englisch -> US
separate-uncertainty,  % Unsicherheiten seperat angeben (mit ±)
]{siunitx}
\usepackage{physics}  % Erstellung von Gleichungen vereinfachen

\usepackage{graphicx}  % Bilder einbinden \includegraphics{Pfad/zur/Datei(ohne Dateiendung)}

% Gestaltung
\usepackage{booktabs}  % schönere Tabellen
\usepackage[toc]{multitoc}  % mehrspaltiges Inhaltsverzeichnis
\usepackage{csquotes}  % Anführungszeichen mit \enquote
\usepackage{caption}  % Anpassung der Bildunterschriften, Tabellenüberschriften
\usepackage{subcaption}  % Unterabbildungen, Untertabellen, …
\usepackage{enumitem}  % Listen anpassen
\setlist{itemsep=-10pt}  % Abstände zwischen Listenpunkten verringern

% Manipulation des Seitenstils
\usepackage[headtopline = .5pt]{scrlayer-scrpage}

% Bibliographie
\usepackage[backend=biber]{biblatex}
\addbibresource{bibliography.bib}

% SI-Einheiten darstellen
\usepackage{siunitx}

% Kopf-/Fußzeilen setzen
\pagestyle{scrheadings}  % Stil für die Seite setzen
\clearmainofpairofpagestyles  % Stil zurücksetzen, um ihn neu zu definieren
\automark{section}  % Abschnittsnamen als Seitenbeschriftung verwenden
\ofoot{\pagemark}  % Seitenzahl außen in Fußzeile
\ihead{\headmark}  % Seitenbeschriftung mittig in Kopfzeile

\usepackage[hidelinks]{hyperref}  % Links und weitere PDF-Features

% TODO: Titel und Autor, … festlegen
\newcommand*{\titel}{Spectroscopy of Gamma Rays}
\newcommand*{\autor}{Sebastian Thiede, Alexander Lettau}
\newcommand*{\abk}{GA}
\newcommand*{\betreuer}{Yingjie Chu}
\newcommand*{\messung}{13.01.2022 \& 20.01.2022}
\newcommand*{\ort}{ASB/K07}

\hypersetup{pdfauthor={\autor}, pdftitle={\titel}}  % PDF-Metadaten setzen

% automatischen Titel konfigurieren
\titlehead{Praktikum des IKTP \abk \hfill TU Dresden}
\subject{Versuchsprotokoll}
\title{\titel}
\author{\autor}
\date{\begin{tabular}{ll}
Protokoll: & \today\\
Messung: & \messung\\
Ort: & \ort\\
Betreuer: & \betreuer\end{tabular}}

% -- Ende Präambel

\begin{document}
\begin{titlepage}
\maketitle  % Titel setzen
\tableofcontents  % Inhaltsverzeichnis setzen
\end{titlepage}

% ----- DOKUMENT ANFANG -----

\section{Theory}

\subsection{Gamma Decay}

Alpha and beta decay usually give energy to the atomic core, which leads to excited nuclear staes.
Deexcitation of the core than leads to the emission of a high energetic photon, the gamma particle, with a specific energy.
These photons have an energy, depending on the core, in the range from $100$ kev to around $8$MeV.
The emission of the photon follows transition rules, depending on the Energy of the photon $E_{\gamma}$, angular momentum $L$ and the change of parity $\Delta \pi$.
Initial ($i$) and final ($f$) state have to follow the equation $E_{\gamma} = E_i - E_f$.
Depending on $L$ and $\Delta \pi$ we differentiate further between:
\begin{itemize}
\item $\Delta \pi (EL) = (-1)^L $ means electric multipole radiation ($EL$)
\item $\Delta \pi (EL) = (-1)^{L+1} $ means magnetic multipole radiation ($ML$)
\end{itemize}

\subsection{Interactions of Photons with Matter}

Photon interaction with matter is mainly dependant on the energy of the photons and the electric charge of the material it traverses.
In generel there are three interactions, which are:
\begin{itemize}
\item photoeffect
\item compton scattering
\item pairproduction
\end{itemize}
In the following section we will give a brief overview over these interactions.

\begin{figure}[ht]
	\centering
    \includegraphics[width=0.85\textwidth]{../plots/interactions.png}
	\caption{Dominating interactions of photons in matter}
	\label{interactions}
\end{figure}

\subsubsection{Photoeffect}

Photoeffect means the removal of electrons from the atomic shell, by absorbing a photon.
The energy of this electron van than be calculated as:
\begin{equation}
E = E_{\gamma} - U_B
\end{equation}
With the photon energy $E_{\gamma}$ and the binding energy of the material $U_B$.
In this experiment we will use a HPGe (high purity germanium) detector, where the photoeffect is dominated by the inner photelectric effect, where this happens inside a semiconductor.
In this case, an electron-hole-pair is created.
By applying a voltage on the detector, the electron and the hole are seperated and create a current, depending on the transfered energy.
In order for this to take place, the photon must have an energy higher than the binding energy of the material.
Because of this, we want detector material with a low binding energy for detectors with good resolution.
For example Ge has a binding energy of $U_B = \SI{0.67}{\electronvolt} $.
This is low enough, that we have to take thermal noise, created in the detector, into consideration, which means that we have to cool down the detector with liquid nitrogen.

Another detector we will use, is a scitnilator made of NaI.
Because of the lower charge of NaI, the outer photoelectric effect is dominating.
To detector this we use a photomultiplier (see Fig. \ref{theorie_PEV}).
Here we apply a voltage on the created photoelectrons, to target them on a dynode.
There they create new electrons and we apply again a voltage.
This is done a few times and we get a electric current, depending on the energy of the initial photon.

\begin{figure}[ht]
	\centering
    \includegraphics[width=0.85\textwidth]{../plots/Photomultiplier_schema_de.png}
	\caption{Photomultiplier \cite{Bild_Photomultiplier}}
	\label{theorie_PEV}
\end{figure}

Since the photoeffect leads to full transmission of the photonenergy, it leads to the creating of so called full energy peaks.
These are high peaks in the gamma spectrum of the detector.
By knowing the energy of gamma photons from the source, we can identify these peaks in the detector and find a curve, which shows the relatiion of detector channels and energy.
This is done during a detector calibration.

\subsubsection{Compton-Streuung}

Compton scattering is the eleastic scattering of a photon with a charged particle, usally an electron.
The energy of a photon scattered on an electron can be calculated as:
\begin{gather}
    E'_{\gamma}(\varphi) = \frac{E_{\gamma}}{1 + \frac{E_{\gamma}}{m_{e} c^{2}} (1 - \cos (\varphi))}
\end{gather}
With the scattered angle $\varphi$, (see Fig. \ref{theorie_Compton_Streuuung}), the initial energy of the poton $E_{\gamma}$ and the electron mass $m_e$

\begin{figure}[ht]
	\centering
    \includegraphics[width=0.85\textwidth]{../plots/The-geometry-of-Compton-scattering-showing-the-directions-of-the-scattered-photon-and.png}
	\caption{Schematic of compton scattering \cite{Bild_Compton_Streuung}}
	\label{theorie_Compton_Streuuung}
\end{figure}
Comtpon effect usually leads to 3 singificant structures in the detector spectrum:
\begin{itemize}
\item The angle depandance of the transmitted energy, leads to a continous amount of possible energies. In the spectrum, this is called the compton continuum.
\item Scattered particles with an angle of $180^{\circ}$ will gain maximum energy transmission. This creates a peak on the right edge of the comtpon continuum.
\item On the other edge of the continuum we get the backscatter peak, which comes from photons that are scattered backwards in the detector, before they can transmit all energy.
\end{itemize}

\subsubsection{Pairproduction}

Pair production is the creation of a particle-antiparticle-pair in the electric field of atomic shell or core.
For this process to happen, incoming photons need to have an energy of at least two times the mass of the created particle.
In the case of an electron positron pair, the photons need to have an energy of at least $2 \cdot m_e = 1022 \text{keV}$.
For all elements, pairproduction only dominates for energies above $5 \text{MeV}$ which is out of range of energies in this lab course.

\input{hpge.tex}
\section{Calibration of the NaI Scintilator}

Calibration of the NaI-scintilator was done analogous to the HPGE.
We used the same probes, exept for a new $^{22}$Na source, instead of $^{152}$Eu for another full energy peak analysis.
The probes were changed, because of the lower resolution fg the scintilator.

For this measurement we took a new measurement of the background radiation with the scintilator (see Fig. \ref{back_nai}.

\begin{figure}[h]
  \includegraphics[width=\linewidth]{pictures/szin_background.png}
  \caption{Background radiation measured with the NaI scintilator}
  \label{nai_untergrund}
\end{figure}

This background was again substracted from every measured spectrum.
The measured spectra can be seen in Fig. \ref{szin}.
Just from looking at the spectra, we can already see that this scintilator has a much lower resolution.
The energy of the peaks were again taken from (iaea einfügen).
One special case here is with the $^{22}$Na sample.
In this spectrum we can't see any good gamma peaks.
The one we used, with an energy of $511$ keV has it's origin in photons that are created by annihilation of electrons und positrons, which are created by beta decay of $^{22}$N.

\begin{figure}[h]
\begin{subfigure}{.5\textwidth}
  \centering
  \includegraphics[width=.9\linewidth]{pictures/szin_co.png}
  \caption{$^{60}\text{Co}$}
\end{subfigure}%
\begin{subfigure}{.5\textwidth}
  \centering
  \includegraphics[width=.9\linewidth]{pictures/szin_cs.png}
  \caption{$^{137}\text{Cs}$}
\end{subfigure}%
 \vskip\baselineskip
\begin{subfigure}{.5\textwidth}
  \centering
  \includegraphics[width=.9\linewidth]{pictures/szin_bi.png}
  \caption{$^{207}\text{Ba}$}
\end{subfigure}%
\begin{subfigure}{.5\textwidth}
  \centering
  \includegraphics[width=.9\linewidth]{pictures/szin_na.png}
  \caption{$^{22}\text{Na}$}
\end{subfigure}%
\caption{gamma spectra of the measured sources}
\label{szin}
\end{figure}

We fitted the marked peaks again with a guassian function:
\begin{equation}
N(K) = a \cdot exp \left( - \frac{(K - K_{0}^{2}}{2 \sigma ^{2}} \right)
\end {equation}
Where $N$ are the counts and $K$ are the channels.
This gave us channels correspondig to a energy and the FHWM as measure of uncertainty.
The results can be seen in Tab, \ref{szin_peaks}.

\begin{table}[h]
\centering
\begin{tabular}{c |c | c |c}
\hline
Isotop & $E_{\gamma}$ [keV]  & Channel & FWHM \\
\hline
$^{137}Cs$ & $661.7$ & 383 & 29 \\
$^{60}Co$ & $1173.2$ & 664 & 27 \\
$^{60}Co$ & $1332.5$ & 753 & 39 \\
$^{207}Bi$ & $569.7$   & 331 & 27 \\
$^{207}Bi$ & $1063.7$ & 605 & 38 \\
$^{22}Na$ & $511.0$   & 298 & 23 \\
\hline
\end{tabular}
\caption{NaI full energy peaks}
\label{szin_peaks}
\end{table}
By plotting the enrgies over the channels we could again do a linear fit to find the relationship between these, for the NaI scintilator.
The results can be seen in Fig. \ref{szin_kali}

\begin{figure}[h]
  \includegraphics[width=\linewidth]{pictures/szin_kali.png}
  \caption{Calibration curve of the NaI scintilator}
  \label{szin_kali}
\end{figure}
Fitparameters were calculated to a slope of $m = 1.8084 \pm 0.0041$ and an intercept of $n = -29.6 \pm 2.2$
Putting this all together gives us for the scintilator the calibration curve:
\begin{equation}
E(K) = [(1.8084 \pm 0.0041) \cdot K - (29.6 \pm 2.2)]  \text{keV}
\end {equation}

\clearpage

\subsection{Energy Resolution}

With the calibration curve, we now looked into the energy resolution of the scintilator.
Just like for the HGPE detector, the resolution should be dependant on:
\begin{itemize}
\item an electronic noise term $a$
\item a term for statistical fluctioations $b$
\item a term for inhomgenieties in the detector $c$
\end{itemize}
For the NaI detector, the inhomogeneties should be of lesser concern for the uncertainty, which means that we will neglect $c$.
This modifies the theoretical function to:
\begin{equation}
\frac{\Delta E}{E} = \frac{\sqrt{a + bE }}{E}
\label{szin_resolution}
\end{equation}
As uncertainty for the energy ($\Delta E$) we used again the FHWM calculated in Tab. \ref{szin_peaks}.
Plotting $\frac{\Delta E}{E}$ und fitting according to Eq. \ref{szin_resolution} gave us Fig. \ref{szin_res}.

\begin{figure}[h]
  \includegraphics[width=\linewidth]{pictures/szin_res.png}
  \caption{Energy reslution of the NaI scintilator}
  \label{szin_res}
\end{figure}
With the fitting contsants: $a = 118 \pm 16$ and $b = 0.96 \pm 0.14$.
In Fig. \ref{szin_res} we can see, that the points have a pretty high deviation from the fit, yet in generell the function seems to work with the data.
This might have to do with the in generel high deviation of this scintilator.
Some more points for the fit should have made the result better.

\clearpage

\input{fukushima.tex}
\section{Unknown Samples}
In this part we got 5 samples of ashe from leaves and pines, from eastern europe.
Of these samples, we took the gamma spectrum with the HPGE detector and compared them with a refference sample.
The sample contains radioactive $^{137}$Cs which is supossed to be contained in the ashe, because of the Chernobyl nuclear accident.
In the following we compare the acitivty of $^{137}$Cs from the ashe with the refference.
From all spectra we substracted the background radiation.

In Fig. \ref{cher_samples} we can see the spectrum of the refference and the samples.
\begin{figure}[h]
\begin{subfigure}{.5\textwidth}
  \centering
  \includegraphics[width=.9\linewidth]{pictures/cher_ref.png}
  \caption{Reference sample}
\end{subfigure}%
\begin{subfigure}{.5\textwidth}
  \centering
  \includegraphics[width=.9\linewidth]{pictures/PP34R1_1.png}
  \caption{Sample $PP34R1\_1$}
\end{subfigure}%
 \vskip\baselineskip
\begin{subfigure}{.5\textwidth}
  \centering
  \includegraphics[width=.9\linewidth]{pictures/PP5PR1_2.png}
  \caption{Sample $PP5PR1\_2$}
\end{subfigure}%
\begin{subfigure}{.5\textwidth}
  \centering
  \includegraphics[width=.9\linewidth]{pictures/PP63R1_3.png}
  \caption{Sample $PP63R1\_3$}
\end{subfigure}%
 \vskip\baselineskip
\begin{subfigure}{.5\textwidth}
  \centering
  \includegraphics[width=.9\linewidth]{pictures/PP84R2_4.png}
  \caption{Sample $PP84R2\_4$}
\end{subfigure}%
\begin{subfigure}{.5\textwidth}
  \centering
  \includegraphics[width=.9\linewidth]{pictures/PPACR2_5.png}
  \caption{Sample $PPACR2\_5$}
\end{subfigure}%
\caption{gamma spectra of the measured samples}
\label{cher_samples}
\end{figure}
Measurement time was around 3h.
In every samples we can clearly  see a  $^{137}$Cs peak, together with smaller peaks.
Some of these smaller peaks might come from natural occuring isotopes, while some might result from the Chernobyl accident too.
For this analysis we will concentrate on $^{137}$Cs.

From the peaks we calculated the measured acitivty again as:
\begin{equation}
A_i = \frac{N_{obs}}{t_{live} \eta (E_i) P}
\end{equation}
The observed counts $N_{obs}$ were this time calculated as:
\begin{equation}
N_{obs} = \frac{I \sigma \sqrt{2 \cdot \pi}}{b}
\end{equation}
With the binwidth $b = 0.4$ to compensate the change of binwidth on the energy scale.
For the analysis we than took the ration $\frac{A_i}{A_0}$ with the activity of the reference $A_0$ and the activity of the samples $A_i$.
The results can be seen in Tab. \ref{cher_results}.
The uncertainties were calculated from the uncertainty of $N_{obs}$ and $\eta$.

\begin{table}[h]
\centering
\begin{tabular}{c |c | c }
\hline
Sample & A [Bq] & $\frac{A}{A_0}$ \\
\hline
Reference & $8.1 \pm 1.2   $ & $1$ \\
$PP34R1\_1  $& $113.7 \pm 1.3$ & $14.1 \pm 4.3$ \\
$PP5PR1\_2  $& $114.4 \pm 1.3$ & $14.2 \pm 4.4$ \\
$PP63R1\_3  $& $106.6 \pm 1.3$ & $13.2 \pm 4.0$ \\
$PP84R2\_4  $& $36.2 \pm 1.3  $ & $4.5 \pm 1.4$ \\
$PPACR2\_5 $& $14.4 \pm 1.3  $ & $1.8 \pm 1.4$ \\
\hline
\end{tabular}
\label{cher_results}
\end{table}
Every measured sample has a higher activity than the reference sample.
The lowest acitivity is with sample $PPACR2\_5$, which has nearly double the acitivity of the reference source.
Three of the samples ($PP34R1\_1, PP5PR1\_2 \text{and} PP63R1\_3$) have an activity more than ten times higher, which could mean, that these contain a considerable amount of radioactive  $^{137}$Cs.

\section{Attenuation}

In this part of the lab course, we investigated the attenuation of radiation from $^{137}$Cs through copper.
The attenuation of radiation can be calculated with:
\begin{equation}
I(d) = I_0 e^{\frac{\mu}{\rho} d  }
\end {equation}
Where $I$ is the attenuated quantity, $I_0$ is this quantity without attenuatiuon, $\frac{\mu}{\rho}$ is the mass attenuation coeficient and $d$ is the lenght over which the attenuation takes place.
To measure these, we took the spectra of $^{137}$Cs, shielded through multiple layers of copper.
Natural occuring copper only has stable isotopes, which means that this shielding shouldn't give much extra background radiation
The measured spectra can be seen in Fig. \ref{attenuation_spectra}.
\begin{figure}[h]
\begin{subfigure}{.5\textwidth}
  \centering
  \includegraphics[width=.9\linewidth]{../Plots/attenuation_0.png}
  \caption{no copper}
\end{subfigure}%
\begin{subfigure}{.5\textwidth}
  \centering
  \includegraphics[width=.9\linewidth]{../Plots/attenuation_1.png}
  \caption{4 mm of copper}
\end{subfigure}%
 \vskip\baselineskip
\begin{subfigure}{.5\textwidth}
  \centering
  \includegraphics[width=.9\linewidth]{../Plots/attenuation_2.png}
  \caption{8 mm of copper}
\end{subfigure}%
\begin{subfigure}{.5\textwidth}
  \centering
  \includegraphics[width=.9\linewidth]{../Plots/attenuation_3.png}
  \caption{4 mm of copper}
\end{subfigure}%
 \vskip\baselineskip
\begin{subfigure}{.5\textwidth}
  \centering
  \includegraphics[width=.9\linewidth]{../Plots/attenuation_4.png}
  \caption{16 mm of copper}
\end{subfigure}%
\caption{Spectra of $^{137}$Cs with copper shielding}
\label{attenuation_spectra}
\end{figure}
Time for measurement was adjusted, that every peak has at least a maximum counting of 1000 events, to give us a better statistic.
Background radiation was again substracted from every spectrum.

We can see, that even the lowest shielding with 4 mm copper was enough to remove the beta-peak on the far left of the spectrum.
This can easily be explained with the low range of electrons in matter and was expected.
As attenuated quanitty, we used the full energy peak to calculate the counts per second from this peak.
This was calculated gain with Eq. \ref{gauss_area}.
The results can be seen in Table \ref{attenuation_values}.
\begin{table}[h]
\centering
\begin{tabular}{c |c }
\hline
Counts per second [s$^{-1}$] & thickness [mm] \\
\hline
334.1 & 0 \\
263.7 & 4 \\
208.6 & 8 \\
161.7 & 12 \\
127.1 & 16 \\
\hline
\end{tabular}
\caption{Attenuation through copper}
\label{attenuation_values}
\end{table}
Plotting this data enabled us to fit the values according to the function:
\begin{equation}
I(d) = I_0 e^{a \cdot d}
\end {equation}
With $a = \frac{\mu}{\rho}$.
Given the density of copper at $300$K ($8.96 \text{gcm}^{-3}$) made it possible to calculate the attenuation coefficient of copper from this fit-parameter.
The data and fit can be seen in Fig. \ref{attenuation_fit}
\begin{figure}[h]
  \includegraphics[width=\linewidth]{../Plots/attenuation.png}
  \caption{Data and fit of the attenuation}
  \label{attenuation_fit}
\end{figure}
We can directly see, that the data are well approximated through the exponential function.
The fit parameter $a$ was calculated to $a = (60.17 \pm 0.26)$.
We can now calculate the attenuation coefficient $(\mu = \frac{a}{\rho}$), which gives us:

$\mu = (6.715 \pm 0.029) \cdot 10^{-2}$ $\text{cm}^2 \text{g}^{-1}$

As a literature value for comparision, we got $\mu = 7.3 \cdot 10^{-2}\text{cm}^2 \text{g}^{-1}$ (Quelle xcom einfügen).
\clearpage

\section{Diskussion}

Alle Aufgabenteile konnten erfolgreich abgeschlossen werden.

Im Niedrigenergiebereich wurden fünf verschiedene Proben auf ihre zusammensetzung untersucht.
Dabei konnten in der 153BeO die Teilchen $^{16}$O$^{1-}$, $^{9}$Be$^{16}$O$^{1-}$ und $^{10}$Be$^{18}$O$^{1-}$ nachgewiesen werden.
In den anderen Proben konnten diverse andere Teilchen, darunter bereits erwähnte mit anderen Nukliden sowie AlO gefunden werden.
Zusätzlich wurden einige unbekannte Stoffe gefunden, die wir in diesem Praktikum nicht genauer identifizieren konnten.
Mit einer genauerem Untersuchung der Probe bzw. genaueren Wissen über potentiell vorkommende Elemente sollte das aber auch möglich sein.

Im Hochenergiebereich wurde nur eine Probe untersucht.
Bei dieser wurden die Nuklide $^{9}$Be, $^{10}$Be, $^{12}$C, $^{16}$O und $^{26}$Al in verschiedenen Ladungszuständen gefunden.
Hier ist teilweise eine recht große Ungenauigkeit zwischen berechneter Masse und dem Nuklid vorhanden.
Dies lässt sich zumindest teilweise auf die unbekannten Ungenauigkeiten der Messung zurückführen.
Möglicherweise wurden hier auch einzelne Ionen fehlgedeutet.
Mit Kenntnis der Ungenauigkeiten könnte dies genauer werden.

Weiterhin wurde die Detektion und Identifikation von $^{10}\text{Be}$ in einer Gasgefüllten Ionisationsdetektor vorgenommen.
Eine Trennung von den meisten unerwünschten Nukliden die nach dem Beschleuniger auftreten erfolgt durch einen Ablenkmagneten und einen ESA.
Die Identifikation ist trotz dem potentiellen Auftreten des Isobars $^{10}\text{B}$ möglich.
Es ist sogar möglich das Isobar quantitativ im Spektrum von $^{10}\text{Be}$ zu trennen, womit grundsätzlich eine genaue Bestimmung des Anteils des radioaktiven Isotops $^{10}\text{Be}$ möglich ist, und damit die eigentliche Altersbestimmung.

Die Anteilsbestimmung wurde im letzten Versuchsteil vorgenommen.
Die Konzentration des Radionuklides $^{10}$Be wurde dazu in vier verschiedenen Proben berechnet.
Die Ergebnisse dafür sind in Tab. \ref{dis_con} zu sehen.
\begin{table}[h]
\centering
\caption{Gemessene Werte zur Berechnung der $^{10}$Be Konzentration in den Proben.}
\begin{tabular}{|c |c| c|}
\hline
Probe& $\frac{\text{Atome}}{\si{\gram}}$ & mittlere Tiefe [cm] \\
\hline
KY13 &  $(\num{1.115} \pm \num{0.050}) \cdot 10^{6} $ & 47,5\\
KY14 &  $(\num{9.61} \pm \num{0.44}) \cdot 10^{5} $ & 57,5 \\
KY16 &  $(\num{7.66} \pm \num{0.36}) \cdot 10^{5} $ & 92,5\\
KY17 &  $(\num{6.23} \pm \num{0.33}) \cdot 10^{5} $ & 112,5\\
\hline
\end{tabular}
\label{dis_con}
\end{table}
Wir konnten feststellen, dass die Konzentration scheinbar linear mit zunehmender Tiefe abnimmt.
Dies haben wir so erklärt, dass $^{10}$Be durch Ereignisse mit kosmischer Strahlung entsteht und tieferliegende Proben aufgrund ihres Alters daher weniger $^{10}$Be enthalten müssen.
Nach dieser Erklärung sollte eigentlich ein exponentieller Abfall stattfinden, was wir aber weder ausschließen noch bestätigen können.
Weitere Proben zur Messung könnten diese Frage aber beantworten.
Eine Verfälschung der Proben durch hochenergetische Strahlung aus anderen Quellen ist potentiell möglich, würde aber nichts am abnehmen der Konzentration ändern.
Eine Neutronenquelle im inneren der Probe würde das Ergebnis entsprechend ändern können, ist aber unwahrscheinlich.


\nocite{*} % alle resourcen auflisten
\printbibliography

% ----- DOKUMENT ENDE -----

\end{document}
