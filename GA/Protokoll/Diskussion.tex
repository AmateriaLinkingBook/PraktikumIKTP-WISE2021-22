\section{Discussion}

\subsection{Calibration of the HPGE Detector}

The calibration of the HPGE detector gave us an calibration curve of:
\[
 $E(K) = [(0.40090 \pm 0.00020) \cdot K + (0.59 \pm 0.43)]  \text{keV}$ 
\]
In the fit, we saw that this curve is a good approximation for the measured points.
The dominating uncertainty comes from the calculated intercept.
More points for the fit, preferably in the low energy range, might reduce this further.
By looking at the energy resolution, we found that the dominating uncertainty of the detector is caused ba electronic noise, with a value of $a = 47 pm 5$.
The term for statistical fluctuations has a value of $b = -3.6 \pm 1.7$ and the term for inhomogeneties a value of $3.6 \pm 1.9$.
It is noticable that the uncertainty of these values is pretty high, even though our measured points are nicely fit with the theoretical function.
This might be for numerical reasons, but this might also be reducable by getting more measured points for the fit.
The peak efficiency of the detector was calculated by fitting as:
\[
\eta = (0.755 \pm 0.014) \cdot E^{-(0.7174  \pm 0.0035)}
\]
With an uncertainty that is in the range of those from the measured values.
The theoretical function was again a good approximation for the data.

\subsection{Calibration of the Nai Scintilator}

Calibration of the NaI scintilator gave us a calibration curve of
\[
E(K) = [(1.8084 \pm 0.0041) \cdot K - (29.6 \pm 2.2)]  \text{keV}
\]
The uncertainty here is higher than for the HGPE, which was expected for technical reasons.
Looking at the curve (see Fig. \ref{szin_kali}) we see, that the energy of the points was quite concentrated in the middle of the energy range.
A better distribution, with points in the lower and higher range, should have reduced the uncertainties.
For the energy resolution, we got again a dominance of electronic noise, with $a = 118 \pm 16$ while the statistixal terme had a value of $b = 0.96 \pm 0.14$ which was even lower than for the HPGE.
Yet by looking at Fig. \ref{szin_res} we see that the points are not that good approximated.
The NaI scintilator has in generell a lower resolution.
Because of this, using sources with a higher activity, or taking longer measurements, could have given better results.

\subsection{Fukushima Sample}
With the given sample we calculated the Fukushima nuclear disaster (11.03.2011) to the 14.03.2010.
The main source for this one year difference should be found in the abundance ratio of $\frac{^{134}Cs}{^{137}Cs}$.
This was given to us as $\frac{6 \%}{68\%}$ and we can't really verify this numbers, but we asume that the abundance of two isotopes during a nuclear meltdown can only be given with a high uncertainty.
More measurements or complex calculations might be needed for better results on this topic.

\subsection{Unknown Samples}

In the given samples we found visible amounts of $^{137}$Cs in the spectra.
The acivities of three of the samples were more than ten times higher than the reference. 
Potential uncertainties might be reduced by longer measurement times.
For further investigation, one might look into the acitivty through other isotopes, created by the accident, that are contained in the sample.
Yet we can see in Fig. that the acitivtx of thoser is at least one order of magnitude lower than  $^{137}$Cs.

\subsection{Attenuation of Gamma Rays}

The attenuation coefficient of Copper could be measured as $\mu = (6.715 \pm 0.029) \cdot 10^{-2}$ $\text{cm}^2 \text{g}^{-1}$, compared to a literature value of  $\mu = 7.3 \cdot 10^{-2}\text{cm}^2 \text{g}^{-1}$-
One reason why our value is too low might be, that we ignored potential background radiation coming from the copper used to shield the source.
Measuring the background for every new copper layer could have gicen a better result.
Using longer times for the measurement should reduce potential uncertainties too.
We also used the NaI scintilator, which has in generell lower accuarcy.
Using the HPGE should have given better results too.

