\section{Diskussion}

Alle Aufgabenteile konnten erfolgreich abgeschlossen werden.

Im Niedrigenergiebereich wurden fünf verschiedene Proben auf ihre zusammensetzung untersucht.
Dabei konnten in der 153BeO die Teilchen $^{16}$O$^{1-}$, $^{9}$Be$^{16}$O$^{1-}$ und $^{10}$Be$^{18}$O$^{1-}$ nachgewiesen werden.
In den anderen Proben konnten diverse andere Teilchen, darunter bereits erwähnte mit anderen Nukliden sowie AlO gefunden werden.
Zusätzlich wurden einige unbekannte Stoffe gefunden, die wir in diesem Praktikum nicht genauer identifizieren konnten.
Mit einer genauerem Untersuchung der Probe bzw. genaueren Wissen über potentiell vorkommende Elemente sollte das aber auch möglich sein.

Im Hochenergiebereich wurde nur eine Probe untersucht.
Bei dieser wurden die Nuklide $^{9}$Be, $^{10}$Be, $^{12}$C, $^{16}$O und $^{26}$Al in verschiedenen Ladungszuständen gefunden.
Hier ist teilweise eine recht große Ungenauigkeit zwischen berechneter Masse und dem Nuklid vorhanden.
Dies lässt sich zumindest teilweise auf die unbekannten Ungenauigkeiten der Messung zurückführen.
Möglicherweise wurden hier auch einzelne Ionen fehlgedeutet.
Mit Kenntnis der Ungenauigkeiten könnte dies genauer werden.

Weiterhin wurde die Detektion und Identifikation von $^{10}\text{Be}$ in einer Gasgefüllten Ionisationsdetektor vorgenommen.
Eine Trennung von den meisten unerwünschten Nukliden die nach dem Beschleuniger auftreten erfolgt durch einen Ablenkmagneten und einen ESA.
Die Identifikation ist trotz dem potentiellen Auftreten des Isobars $^{10}\text{B}$ möglich.
Es ist sogar möglich das Isobar quantitativ im Spektrum von $^{10}\text{Be}$ zu trennen, womit grundsätzlich eine genaue Bestimmung des Anteils des radioaktiven Isotops $^{10}\text{Be}$ möglich ist, und damit die eigentliche Altersbestimmung.

Die Anteilsbestimmung wurde im letzten Versuchsteil vorgenommen.
Die Konzentration des Radionuklides $^{10}$Be wurde dazu in vier verschiedenen Proben berechnet.
Die Ergebnisse dafür sind in Tab. \ref{dis_con} zu sehen.
\begin{table}[h]
\centering
\caption{Gemessene Werte zur Berechnung der $^{10}$Be Konzentration in den Proben.}
\begin{tabular}{|c |c| c|}
\hline
Probe& $\frac{\text{Atome}}{\si{\gram}}$ & mittlere Tiefe [cm] \\
\hline
KY13 &  $(\num{1.115} \pm \num{0.050}) \cdot 10^{6} $ & 47,5\\
KY14 &  $(\num{9.61} \pm \num{0.44}) \cdot 10^{5} $ & 57,5 \\
KY16 &  $(\num{7.66} \pm \num{0.36}) \cdot 10^{5} $ & 92,5\\
KY17 &  $(\num{6.23} \pm \num{0.33}) \cdot 10^{5} $ & 112,5\\
\hline
\end{tabular}
\label{dis_con}
\end{table}
Wir konnten feststellen, dass die Konzentration scheinbar linear mit zunehmender Tiefe abnimmt.
Dies haben wir so erklärt, dass $^{10}$Be durch Ereignisse mit kosmischer Strahlung entsteht und tieferliegende Proben aufgrund ihres Alters daher weniger $^{10}$Be enthalten müssen.
Nach dieser Erklärung sollte eigentlich ein exponentieller Abfall stattfinden, was wir aber weder ausschließen noch bestätigen können.
Weitere Proben zur Messung könnten diese Frage aber beantworten.
Eine Verfälschung der Proben durch hochenergetische Strahlung aus anderen Quellen ist potentiell möglich, würde aber nichts am abnehmen der Konzentration ändern.
Eine Neutronenquelle im inneren der Probe würde das Ergebnis entsprechend ändern können, ist aber unwahrscheinlich.
