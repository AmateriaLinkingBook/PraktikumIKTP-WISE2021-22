\section{Aufgabenstellung}
Die Beschleunigermas­senspektrometrie (AMS) ist ein wichtiges Werkzeug zur Trennung und Detektierung von Nukliden.
Eine der häufigsten Anwendungen ist die Altersbestimmung von Proben aus der Natur durch Messung der Konzentration verschiedener Nuklide, die auf der Erdoberfläche durch kosmische Strahlung entstehen.
Im Gegensatz zu anderen Verfahren der Massenspektrometrie erreicht man bei der AMS eine sehr gute Unterdrückung atomarer und molekularer Isobare.
Die Messzeiten sind relativ kurz und es lassen sich auch kleine Proben untersuchen.

In diesem Versuch wurde der Beschleuniger DREAMS (DREsden AMS) vorgestellt.
Die generelle Handhabung wurde anhand folgender praktischer Aufgaben kennengelernt:
\begin{itemize}
  \item Inbetriebnahme eine Sputter-Ionenquelle und Erzeugung negativer Ionen
  \item Strahlentransport eines Isotopenpaares durch den Beschleuniger
  \item Kalibrierung der Verstärkung der gasgefüllten Ionistationskammer
  \item Aufnahme von Messwerten in der gasgefüllten Ionisationskammer
\end{itemize}

Zur Auswertung sind folgende Aufgaben zu bearbeiten:
\begin{itemize}
  \item Berechnung der Teilchenenergien nach dem Beschleuniger
  \item Plotten des Stromes des Ionenstrahls im Faraday Cup
  \item Abschätzung des Energieverlustes des Ionenstrahls in einer dünnen Folie
  \item Abschätzung des Energieverlustes des Ionenstrahls in Isobutan (Gas in der Ionisationskammer)
  \item Plotten der gemessenen Spektren und Identifikation der Ionen
  \item Berechnung der Konzentration von Radionukliden in einer unbekannten Probe
\end{itemize}
