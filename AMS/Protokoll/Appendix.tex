\section{Spezieszuordnung}
\label{Appendix_Normen}
Mithilfe einer Norm lassen sich die Permutationen von Ionenspezies bestimmen, die die Peaks im Faraday-Cup am besten beschreiben.
Zuerst wurden die gemessenen Magnetstromstärken-Peaks $\vec{P}_{\text{exp}}$ ermittelt.
Aus diesen wurden die Verhältnisse $\vec{r}_{\text{exp}}$ gebildet.
Dann wurde für die möglichen Ionenspezies $\vec{P}_{\text{möglich}}$ in der Probe das Verhältnis $\frac{\sqrt{m_{\text{ion, möglich}}}}{q_{\text{ion, möglich}}}$ gebildet.
Die Ionenspezies wurden nach diesem Verhältnis geordnet und die erwähnten, geordneten Permutationen $\vec{\nu}(k)$ gebildet.
(Nur Permutationen in dem die Reihenfolge der Elemente nicht vertauscht wird, aber Elemente weggelassen werden können.)
Für diese Permutationen wurde dann ebenfalls das Verhältnis $\vec{r}_{\text{möglich}\ \nu_{i}(k)}$, diesmal von $\frac{\sqrt{m_{\text{ion, möglich}}}}{q_{\text{ion, möglich}}}$, erstellt.
(Das ist ein Vektor für jede Permutation.)
Dann wurde mithilfe der euklidischen Norm der \glqq Abstand\grqq{} der Verhältnisse gebildet:
\begin{gather}
    \norm{\vec{r}_{\text{exp}}-\vec{r}_{\text{möglich}\ \nu_{i}(k)}} = \sqrt{\sum_{n}(\vec{r}_{\text{exp}\ n}-\vec{r}_{\text{möglich}\ \nu_{i}(k),n})^{2}}
\end{gather}
(In Anlehnung an die Methode der kleinsten Fehlerquadrate. Man könnte die Wurzel auch weglassen. Da die Wurzelfunktion monoton ist ändert sie die Reihenfolge später nicht.)
Die Permutation mit der kleinsten Norm ist dann gerade die beste Zuordnung der Ionenspezies zu den Peaks.
