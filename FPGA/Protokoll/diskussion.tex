\section{Diskussion}
Wir haben in diesem Versuch gelernt wie man Daten mithilfe eines FPGA verarbeitet.
Anhand von Daten wie sie typischerweise in Kalorimetern am ATLAS-Detektor auftreten, haben wir ein Modul zusammengestellt, welches die Daten digital filtert und das Maximum der gefilterten Daten bestimmt.
Es ist klar geworden, dass sich viele Aufgaben, die man typischerweise in sequentiellen Rechnern implementieren würde, parallelisieren lassen, um auch mit größeren Datenmengen fertig zu werden.
Die Parallelisierung von Prozessen stellt sich zu Beginn als Herausforderung dar, da ein Umdenken weg vom sequentiellen verarbeiten der Daten erfolgen muss.
Auch ist klar, dass FPGA den klassischen Computer trotz der großen Verarbeitungsgeschwindigkeiten nicht ablösen kann.
Tatsächlich ist es möglich eine CPU in einem FPGA zu implementieren.
Diese müssen dann aber Abstriche bei den Taktgeschwindigkeiten machen, und damit auch bei der (sequentiellen) Rechenleistung, da Signallaufzeiten nicht so weit optimierbar sind wie bei herkömmlichen CPUs.

In jedem Fall lässt sich der Versuch als gelungen betrachten.
Wir sind sehr zufrieden gewesen mit der Aufgabenstellung; durch die Einführung über ein physikalisch relevantes Thema stellt sich auch die typische Frage \glqq Wofür brauch ich das\grqq{} nicht.
Wir hoffen in Zukunft weiter mit FPGA arbeiten zu können.
