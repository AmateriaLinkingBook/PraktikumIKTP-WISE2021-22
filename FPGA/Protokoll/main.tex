% Vorlage: https://www.pfsr.de/latex

% -- Anfang Präambel
\documentclass[german,  % Standardmäßig deutsche Eigenarten, englisch -> english
parskip=full,  % Absätze durch Leerzeile trennen
%bibliography=totoc,  % Literatur im Inhaltsverzeichnis (ist unüblich)
%draft,  % TODO: Entwurfsmodus -> entfernen für endgültige Version
]{scrartcl}

\usepackage[utf8]{inputenc}  % Kodierung der Datei
\usepackage[T1]{fontenc}  % Vollen Umfang der Schriftzeichen
\usepackage[ngerman]{babel}  % Sprache auf Deutsch (neue Rechtschreibung)

% Mathematik und Größen
\usepackage{amsmath}
\usepackage[locale=DE,  % deutsche Eigenarten, englisch -> US
separate-uncertainty,  % Unsicherheiten seperat angeben (mit ±)
]{siunitx}
\usepackage{physics}  % Erstellung von Gleichungen vereinfachen

\usepackage{graphicx}  % Bilder einbinden \includegraphics{Pfad/zur/Datei(ohne Dateiendung)}

% Gestaltung
\usepackage{booktabs}  % schönere Tabellen
\usepackage[toc]{multitoc}  % mehrspaltiges Inhaltsverzeichnis
\usepackage{csquotes}  % Anführungszeichen mit \enquote
\usepackage{caption}  % Anpassung der Bildunterschriften, Tabellenüberschriften
\usepackage{subcaption}  % Unterabbildungen, Untertabellen, …
\usepackage{enumitem}  % Listen anpassen
\setlist{itemsep=-10pt}  % Abstände zwischen Listenpunkten verringern

% Manipulation des Seitenstils
\usepackage[headtopline = .5pt]{scrlayer-scrpage}

% Bibliographie
\usepackage[backend=biber]{biblatex}
\addbibresource{bibliography.bib}

% SI-Einheiten darstellen
\usepackage{siunitx}

% Code - Blöcke
\usepackage{listings}
\usepackage[dvipsnames]{xcolor}

% Kopf-/Fußzeilen setzen
\pagestyle{scrheadings}  % Stil für die Seite setzen
\clearmainofpairofpagestyles  % Stil zurücksetzen, um ihn neu zu definieren
\automark{section}  % Abschnittsnamen als Seitenbeschriftung verwenden
\ofoot{\pagemark}  % Seitenzahl außen in Fußzeile
\ihead{\headmark}  % Seitenbeschriftung mittig in Kopfzeile

\usepackage[hidelinks]{hyperref}  % Links und weitere PDF-Features

% TODO: Titel und Autor, … festlegen
\newcommand*{\titel}{Elektronische Detektorauslese und digitale Datenverarbeitung mit FPGA}
\newcommand*{\autor}{Sebastian Thiede, Alexander Lettau}
\newcommand*{\abk}{FP}
\newcommand*{\betreuer}{Johann C. Voigt}
\newcommand*{\messung}{22.11.2021 \& 24.11.2021}
\newcommand*{\ort}{ASB/K07}

\hypersetup{pdfauthor={\autor}, pdftitle={\titel}}  % PDF-Metadaten setzen

% automatischen Titel konfigurieren
\titlehead{Praktikum des IKTP \abk \hfill TU Dresden}
\subject{Versuchsprotokoll}
\title{\titel}
\author{\autor}
\date{\begin{tabular}{ll}
Protokoll: & \today\\
Datum Praktikum: & \messung\\
Ort: & \ort\\
Betreuer: & \betreuer\end{tabular}}

\lstset{ % General setup for the package
    language=VHDL,
    basicstyle=\small\sffamily,
    numbers=left,
    numberstyle=\tiny,
    frame=single,
    tabsize=4,
    columns=fixed,
    showstringspaces=false,
    showtabs=false,
    keepspaces,
    breaklines=true,
    aboveskip=10pt,
    belowskip=10pt,
    commentstyle=\color{ForestGreen},
    keywordstyle=\color{ProcessBlue},
    identifierstyle=\color{Black}
}

\renewcommand\lstlistingname{Quelltext} % Change language of section name

% -- Ende Präambel

\begin{document}
\begin{titlepage}
\maketitle  % Titel setzen
\tableofcontents  % Inhaltsverzeichnis setzen
\end{titlepage}

% ----- DOKUMENT ANFANG -----

\section{Aufgabenstellung}
In diesem Versuch sollen grundlegende Kenntnisse der elektronischen Datenverarbeitung mittels FPGA, sowie die Konfiguration dieser mittels VHDL vermittelt werden.
Dazu sollen folgende Module programmiert, simuliert und zum Teil auf ein vorliegendes FPGA übertragen werden:
\begin{itemize}
\item mehrere einfache Logikschaltungen (Details sind in der Auswertung zu finden)
\item Ein Modul \glqq LoHi Detect\grqq{}
\item Ein Modul \glqq Max Find\grqq{} welches das Maximum des Inputs ausgibt
\item Ein Modul \glqq Filter\grqq{} das einen Wiener Filter zur Unterdrückung des Rauschens im Input implementiert
\item Eine Uhr die \SI{9}{\s} auf einer Digitalanzeige zählt und sich danach zurücksetzt
\item Ein Modul \glqq ssd\grqq{} das einen vierstelligen Input auf einem digitalen Display anzeigt
\end{itemize}
Die vier Module sollten im Anschluss in einem Modul \glqq top\grqq{} zusammengeschlossen werden, welches als Filter für einen elektronischen Impuls, wie er bei einem Kalorimeter auftritt, fungieren soll.

\section{Theorie}

\subsection{FPGA}
Bei Field Programmable Gate Arrays (FPGA) handelt es sich um spezielle Integrierte Schaltkreise (IC), die nach Herstellung vom Benutzer frei konfiguriert werden können.
Dies trennt sie von herkömmlichen Application Specific Integrated Circuits (ASIC) welche nach Herstellung nicht abgeändert werden können.

Vorteile von FPGA, die sie so interessant in der Anwendung machen, sind unter anderem ihre bereits erwähnte Konfigurierbarkeit, gute Parallelisierbarkeit, ein geringes Risiko von Fehldesigns, die schnelle Implementierung neuer Designs und recht niedrige Kosten für geringe Stückzahlen.
Gleichzeitig existieren aber auch diverse Nachteile, wie geringere Taktfrequenzen als ASICS, geringere Logikdichten, höherer Stromverbrauch sowie eine allgemein geringere Strahlenhärte.

FPGA besitzen Eingangs- und Ausgangssignale (I/O-Port).
Die Eingangssignale können mithilfe von Logikblöcken verbunden werden und so ein gewünschtes Ausgangssignal, abhängig vom Eingang, erzeugt werden.
Um die Rekonfigurierbarkeit von FPGA zu ermöglichen, werden \glqq Schalter\grqq{} zwischen den Logikblöcken eingefügt, die durch Anlegen einer Spannung manipuliert werden können.
Ein Beispiel für die Schaltung zwischen Logikblöcken ist in Abb. \ref{schalter} zu sehen.
\begin{figure}[h]
  \includegraphics[width=\linewidth]{../Daten/schalter.png}
  \caption{rekonfigurierbare Schalter zwischen Logigblöcken}
  \label{schalter}
\end{figure}
Um ein FPGA auf gewünschte Art zu konfigurieren, müssen Hardware-Programmiersprachen (\glqq Hardware Description Language\grqq{} oder HDL) verwendet werden.
Bei FPGA stellen diese eine Art \glqq Schaltplan\grqq{} der logischen Verknüpfungen innerhalb des FPGA dar.
Hierbei ist zu beachten, dass bei der Konfiguration auch wirklich Signale vom Eingang zum Ausgang geleitet und dabei mittels logischer Funktionen manipuliert werden.
Aufgrund dessen ist auch die gute Parallelisierbarkeit des FPGA leicht zu verstehen.
Umfang sowie Geschwindigkeit der Konfiguration ist hier jedoch prinzipiell durch Logikdichte sowie Signallaufzeit des FPGA beschränkt.

\subsection{Kalorimeter}
Kalorimeter dienen in der Teilchenphysik zur Messung der Energie einfallender Teilchen.
Die Idee hinter diesen ist es, die einfallenden Teilchen (z. B. Elektronen oder Photonen) vollständig im Kalorimeter zu stoppen und die dabei freigesetzte Energie zu messen.
Um Kalorimeter in der Praxis umzusetzen gibt es verschiedene mögliche Bauarten, auch abhängig davon welche Teilchen gemessen werden sollen.
Im folgenden beschreiben wir ein elektromagnetisches Sampling-Kalorimeter, wie es auch im ATLAS-Detektor verwendet wird.
Ein solches besteht aus einem Verbund von mehreren Absorber- und Ausleseschichten, an die eine Hochspannung geschlossen ist.
Innerhalb der Absorberschichten geben die Teilchen Energie ab, wobei das Absorbermaterial ionisiert wird.
Durch die Hochspannung werden die so freigesetzten Ladungen abgesaugt, wodurch ein elektrischer Impuls entsteht.
In der Theorie bewegen sich hochenergetische Teilchen nahezu mit Lichtgeschwindigkeit durch den Kalorimeter und ionisieren entlang ihres Weges, was zu Dreiecksimpulsen führt.
Das heißt die gemessene Spannung springt auf einen Wert, der abhängig von der Energie des Teilchens ist, und fällt dann linear ab.
In der Praxis gibt es im Wesentlichen drei Störquellen, die den Impuls verzerren und so den relativen Fehler beeinflußen:
\begin{itemize}
\item Eine ist rein stochastischer Natur und verhält sich proportional zu $\frac{1}{\sqrt{E}}$.
Diese entsteht dadurch, dass die Genauigkeit eines Kalorimeters von der Bestimmung der Anzahl ionisierter Teilchen abhängig ist, welche wiederum einer Poisson-Statistik genügt.

\item Desweitern gibt es eine konstante Störquelle durch Inhomgenitäten im Detektor sowie Energieverluste der Teilchen außerhalb der Absorberschichten.

\item In diesen Versuch von besonderem Interesse ist der sog. \glqq Noise-Term\grqq{} der proportional zu $\frac{1}{E}$ ist, wobei $E$ die im Detektor deponierte Energie ist.
Dieser entsteht einerseits durch elektronisches Rauschen und andererseits durch Untergrundstrahlung oder das Pile-Up, das entsteht, wenn eine Überlagerung mehrerer zu detektierender Ereignisse im Detektor stattfindet (siehe zum Beispiel Abb. \ref{pileup}).
\end{itemize}

\begin{figure}[h]
  \includegraphics[width=\linewidth]{../Daten/pileup.png}
  \caption{Beispielhafte Darstellung eines Pile-Ups}
  \label{pileup}
\end{figure}
Für die Analyse der Detektoren ist es wesentlich dieses Rauschen zu unterdrücken und auch allgemeiner die Impulse in eine digital auslesbare Form zu bringen.
Eine Möglichkeit dafür besteht in der Verwendung digitaler Filteralgorithmen, welche z. B. im FPGA implementiert werden können.
Um die Untergrundstrahlung bzw. das Pile-Up, welche zu einem ständigen Anwachsen des Impulses bzw. einem verschieben der Nulllinie führen, zu unterdrücken, wird der Impuls durch eine Kombination von RC- bzw. CR-Schaltungen (R = Widerstand, C = Kondensator) geführt.
Diese Schaltungen \glqq integrieren\grqq{} bzw. \glqq differenzieren\grqq{} den Impuls und wirken gleichzeitig als Frequenzfilter.

Als nächstes wird auf den Impuls eine \glqq digitale Filterung\grqq{} angewandt.
Dabei werden die nun diskreten Spannungswerte mit speziellen, vorher bestimmten Konstanten multipliziert und aufsummiert, was das Rauschen im Impuls unterdrückt (ohne Beweis).
Diese Summe ergibt nun Energiewerte (also die Pulshöhe), was als Ausgabe des FPGA dienen kann.
Damit ergibt sich die Energie zu:
\begin{equation}
E = \sum_{i = 1}^{N} a_i S_i
\end{equation}
Hier sind $a_i$ die Filterkonstanten und $S_i$ die Impulsbeträge.

Da dieser Prozess in einem modernen Kalorimeter jede Sekunde für große Datenmengen gleichzeitig durchgeführt werden muss, sind FPGA wegen der bereits erwähnten Parallelisierbarkeit für diese Filter prädestiniert.
In diesem Versuch soll daher eine digitale Filterung, wie sie im vorherigen Absatz beschrieben wurde, konfiguriert werden.


\section{Versuchsaufbau}
In diesem Versuch wird ein Entwicklungsboard mit Intel/Altera Cyclone III FPGA verwendet (siehe Abb. \ref{board}).
Dieses verfügt neben einem FPGA über einen An/Aus-Schalter, LEDs, vier ansteuerbare Siebensegmentanzeigen, mehrere Schalter sowie eine USB-Verbindung.

\begin{figure}[h]
  \includegraphics[width=\linewidth]{../Daten/board.png}
  \caption{Elvis II Entwicklungsboard}
  \label{board}
\end{figure}

Zur Konfiguration des FPGA werden Programmcodes in der Programmiersprache VHDL geschrieben (Details zu dieser sind in der Auswertung zu finden).
Diese Skripte werden danach auf einem PC simuliert, um Funktionalität zu gewährleisten und eventuelle Fehlerquellen zu finden.
Wenn keine vorhanden sind, wird das Skript auf das FPGA übertragen.
Einzelne Skripte konnten dann direkt auf dem FPGA getestet werden.

\section{Workshop}
In dem Workshop haben wir gelernt wie man VHDL schreibt, wie man VHDL-Dateien simuliert und synthetisiert und wie man synthetisierte Dateien auf einen Hardware FPGA überträgt.

\subsection{Beispiel - NAND-Gatter}
Das erste Beispiel das wir uns angeschaut haben ist ein NAND-Gatter.
Anhand diesem wollen wir die grundlegende Syntax erklären.
Dazu ist hier der verwendete Code:

\lstinputlisting[firstline = 10]{../Daten/example/example.vhd}

Die ersten beiden Zeilen binden aus der Bibliothek \textbf{IEEE} den Datentyp \textbf{std\_logic} ein, der im wesentlichen einen boolean darstellt.
Daraufhin wird in Zeile $4$ eine \textbf{entity} erstellt.
Diese stellen Schnittstellen für Logikbausteine dar.
Es wird mittels der Schlüsselwörter \textbf{in} und \textbf{out} definiert welche Signale als Ein- und Ausgänge genutzt werden, und es wird ihnen ein Typ zugewiesen.
Außerdem können diese Signale initialisiert werden wie z.B. in Zeile $8$.
Das erstellen der \textbf{entity} wird in Zeile $10$ beendet.

In Zeile $12$ wird eine \textbf{architecture} erstellt die die Funktionalität der eben erstellten \textbf{entity} beschreibt.
Darauf wird ein internes \textbf{std\_logic} Signal, auf das von außen nicht zugegriffen werden kann, erstellt und initialisiert.
Im Folgenden wird die eigentliche Logik des NAND-Gatters implementiert.
Dazu wird dem internen Signal das Ergebnis der boolschen AND Verknüpfung zugewiesen und dem Ausgabesignal der negierte Wahrheitswert des internen Signals.
Hierbei ist zu beachten das diese Zuweisung nicht sequenziell passiert, sondern das tatsächliche Verknüpfungen der Signale durch Gatter gemeint sind.
Eine Änderung des Eingangssignals wirk sich damit (bis auf Signallaufzeiten) sofort auf den Ausgang aus.
Es ist daher auch nicht wichtig in welcher Reihenfolge diese Verknüpfungen aufgelistet sind.
In Zeile $19$ wird dann schlussendlich das Beschreiben der \textbf{architecture} beendet.

\subsection{Übung - Logische Gatter}

In der ersten Übung sollten drei LEDs mithilfe von drei Schaltern und einem Knopf nach bestimmten Bedingungen angesteuert werden. Diese lauten:
\begin{itemize}
    \item Sind alle Schalter an soll LED $0$ an sein (sonst aus)
    \item Sind mindestens zwei der Schalter an soll LED $1$ an sein (sonst aus)
    \item Ist Schalter $2$ an und der Knopf nicht gedrückt soll LED $2$ an sein (sonst aus)
\end{itemize}

Dazu haben wir folgende VHDL Datei erstellt:

\lstinputlisting[firstline = 10]{../Daten/gates/gates.vhd}

In der Beschreibung der \textbf{entity} \textbf{gates} sind die drei Schalter und der Knopf als Eingang, sowie die drei LEDs als Ausgang zu erkennen.
Um die Funktionalität zu implementieren wurde in der \textbf{architecture} jedem LED-Ausgangssignal eine entsprechende Logikverknüpfung der Eingangssignale zugewiesen.
So wurde zum Beispiel für die LED $1$ die Signale der Schalter jeweils Paarweise AND verknüpft und die Erbegnisse davon OR verknüpft.

\begin{figure}[ht]
	\centering
    \includegraphics[width=0.98\textwidth]{../Daten/gates.png}
	\caption{Simulation von gates.vhd}
	\label{img_gates}
\end{figure}

In Abb. \ref{img_gates} wurde diese VHDL Datei simuliert.
Man sieht das die gewünschte Funktionalität gegeben ist:
\begin{itemize}
    \item LED $0$ ist nur an wenn alle Schalter an sind
    \item LED $1$ ist an solange mindestens zwei Schalter an sind
    \item LED $2$ ist an wenn Schalter $2$ an und der Knopf nicht gedrückt ist
\end{itemize}

Wir haben diese VHDL-Datei außerdem synthetisiert und dem FPGA aufgespielt um die Funktionalität zu überprüfen.
Es funktionierte einwandfrei.
Ein Bild von einer beispielhaften Schalterstellung ist in Abb. \ref{photo_gates} zu sehen.

\begin{figure}[ht]
	\centering
    \includegraphics[width=0.98\textwidth]{../Daten/Photo_FPGA_gates.png}
	\caption{Photo des FPGA mit aufgespielter synthetisierter Verknüpfung von gates.vhd in beispielhafter Schalterstellung. Da nur Schalter $0$ und $2$ an (und der Knopf nicht gedrückt) sind sollten LED $1$ und $2$ an sein, und sie sind es.}
	\label{photo_gates}
\end{figure}

\subsection{Modul 1 - \textit{LoHi Detect}}

Ab hier wurde an dem Projekt für die Datenverarbeitung am ATLAS Kalorimeter gearbeitet.
In solchen Anwendungen wird üblicherweise mit einer \textit{Clock}, einem Taktsignal gearbeitet.
Da wir hier mit vorgefertigten Daten arbeiten müssen wir ein Startsignal zum Senden der Daten von außerhalb (vom PC) geben.
Dafür verwenden wir einen Knopf.
Um das Senden der Daten nur einmal und in definierter Weise auszulösen müssen wir den Knopfdruck an die Clock koppeln.
Dafür ist das Modul \textit{LoHi Detect} zuständig.
Die Aufgabe bestand darin das ausgehende Signal (sig\_o) dann auf $1$ zu setzen wenn zu einer steigenden Flanke der Clock der Knopf gedrück ist, zur wiederum nächsten steigenden Flanke der Clock sollte es jedoch wieder auf $0$ gesetzt werden.
Im Folgenden ist der Code dieses Moduls zu sehen:

\lstinputlisting[firstline = 10]{../Daten/lohi_detect/lohi_detect.vhd}

Hier wurde ein neues Syntaxelement genutzt: der \textbf{process}.
Dieser tritt innerhalb der \textbf{architecture} auf und wird immer dann ausgeführt wenn eines der Signale die nach dem Schlüsselwort \textbf{process} aufgezählt werden sich ändert.
Im Gegensatz zum Großteil der restlichen VHDL Syntax wird innerhalb eines \textbf{process} sequentiell gearbeitet.
Eine Änderung der Signale findet jedoch erst statt wenn der Prozess beendet ist.
Dadurch sind \textbf{if} Bedingungen und Schleifen möglich.
Intern (beim synthetisieren) werden diese jedoch aufgedröselt, denn im FPGA läuft nichts tatsächlich sequentiell ab.
In dem Code-Beispiel wird also zu jeder steigenden Flanke der Clock dem Ausgangssignal der Wert des Knopfes AND verknüpft mit dem inversen des internen Signal reg zugewiesen.
Dies ist zum ersten Clock-Zyklus eine $1$.
Dem interne Signal reg wiederum wird der Wert des Eingangssignals zugewiesen.
Das verhindert das im nächsten Clock-Zyklus das Eingangssignal auf das Ausgangssignal übergeht; Statdessen wird das Ausgangssignal $0$.
Solange der Knopf gedrück bleibt ändert sich daran nichts.
Wird der Knopf losgelassen wird zur nächsten steigenden Flanke der Clock das interne Signal auf $0$ gesetzt und man befindet sich im Ausgangszustand.

\begin{figure}[ht]
	\centering
    \includegraphics[width=0.98\textwidth]{../Daten/lohi_detect.png}
	\caption{Simulation von lohi\_detect.vhd}
	\label{img_lohi_detect}
\end{figure}

In Abb. \ref{img_lohi_detect} wurde dieses Modul simuliert.
Man sieht das nach Betätigung des Knopfes zur nächsten steigenden Flanke der Clock der Ausgang einen Zyklus lang auf $1$ steht.
Als der Knopf losgelassen wird und das Eingangssignal wieder auf $0$ fällt wird zur nächsten steigenden Flanke der Clock auch das interne Signal reg zurückgesetzt.


\nocite{*} % alle resourcen auflisten
\printbibliography

% ----- DOKUMENT ENDE -----

\end{document}
