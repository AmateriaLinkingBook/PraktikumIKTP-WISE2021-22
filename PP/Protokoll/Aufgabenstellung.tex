\section{Aufgabenstellung}
In diesem Versuch soll die Messung und Auswertung von Daten, die am LHC aus Proton-Proton Kollisionen gewonnen wurden, nachvollzogen werden.
Dafür werden diverse Monte-Carlo-Simulationen, die Daten vom ATLAS-Detektor simulieren, analysiert und bearbeitet.
Konkret wurden folgende Aufgaben bearbeitet:
\begin{itemize}

\item Für den $Z^0 \rightarrow e^+ e^-$ Zerfall sollte die invariante Elektron-Positron-Masse, a) mit Berücksichtigung der Elektronenmasse, b) mit der Näherung $m_e = 0$ bestimmt werden. 

\item Es wurde ein Mystery Datensatz, mit verschiedenen Detektoraufnahmen eines unbekannten Ereignisses, sowie diverse mögliche Ereignisse vorgegeben.
Aus markanten Eigenschaften der Ereignisse (detektierte Teilchen, Anzahl Jets) sollte das Ereigniss identifiziert werden.

\item Die Existenz des Higgs-Bosons soll mithilfe von Monte-Carlo-SImulationen überprüft werden.
Dazu sollen als erstes Cuts, bassierend auf physiaklischen Überlegungen, angesetzt werden, um den Untergrund des Ereignisses zu verringern.
Bassierend auf diesen Cuts werden dann Histogramme von Monte-Carlo-Simulationen erstellt und diese statistisch ausgewertet.
Konkret soll der p-Value berechnet werden.

\item Ausgehend von Monte-Carlo-Simulationen soll nach einem hyptothetischen $Z'$-Boson gesucht werden.
Dafür stehen mehrere Simulationen mit verschiedenen Massen des $Z'$ zur verfügung, die wieder statistisch, mit p-Value Berechnung, ausgewertet werden sollen.

\end{itemize}
