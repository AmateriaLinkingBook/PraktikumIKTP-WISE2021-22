\section{Diskussion}

Alle Versuchsaufgaben konnten abgeschlossen werden.

Die invariante Masse bei $Z$-Zerfällen wurde mithilfe von vier Ereignissen zu $(67 \pm 11)$ GeV bestimmt.
Dies liegt offenbar unter der theoretischen Masse des Z-Bosons von $91$ GeV, was jedoch schon in der Auswertung erklärt wurde.
Eine Verbesserung dieses Wertes sollte sich unter Einbeziehung weiterer Ereignisse erreichen lassen, da vier Ereignisse ja eine recht kleine Statistik liefern.
Es wäre ebenfalls noch möglich systematische Messunsicherheiten zu betrachten. 
Da die Daten jedoch vom ATLAS-Detektor stammen gehen wir davon aus, dass diese Unsicherheiten zumindest für diesen Versuch zu vernachlässigen sind.

Das Ereignis im Mysterydatensatz $12$ wurde mit dem $ZZ$-Ereignis identifiziert.
Da die Bestimmung unserer Ansicht nach eindeutig war, gibt es dazu auch keine Fehlerquellen, auf die wir eingehen könnten.
Eine mögliche wäre, dass die Mindestestenergie für Teilchen zu hoch angesetzt wurde und deshalb relevante Informationen ignoriert wurden.
Diese Mindestenergie wurde von uns jedoch für mehrere Ereignisse variiert, weshalb wir dies als Fehlerquelle ausschließen.
Man könnte das Ergebnis noch verifizieren, indem invariante Massen der Ereignisse bestimmt werden.

Bei der Higgs-Analyse ergab sich ein p-Value von 0.0039, was eine Übereinstimmung der Daten mit dem Standardmodell bedeutet, wie es zu erwarten war.
Nach Konventionen der Teilchenphysik ist dieser Wert jedoch noch nicht groß genug um von einer Entdeckung des Higgs-Bosons innerhalb der Daten zu reden.
Möglicherweise lassen sich noch bessere Cuts setzen, um das Higgs-Signal besser herauszuarbeiten, was den p-Value verbessern sollte.

Ein $Z'$-Boson konnte in den vorliegenden Daten im Massenbereich von $400-2500$ GeV nicht entdeckt werden.
Die nobelpreiswürdige Entdeckung neuer Physik konnte damit in diesem Praktikumsversuch leider nicht erreicht werden. :(
Für eine Masse von $400$ GeV wurde jedoch ein p-Value von 0,32 berechnet, was zumindest als Hinweis gedeutet werden kann und nähere Beschäftigung mit den Daten motiviert.
Um ein zufälliges Rauschen in den Daten auszuschließen, müsste die Messgenauigkeit bzw. die Simulation der Monte-Carlo-Simulationen überprüft werden.
Für eine potentielle Entdeckung oder Ausschließung eines $Z'$ in diesem Massebereich wäre die Analyse weiterer Daten notwendig.
Eine Untersuchung mit genaueren Massebereichen könnte ebenfalls weitere Erkenntnisse bringen.
Nicht ausgeschlossen werden kann, dass die angewandten Cuts nicht optimal waren oder gar ein eventuelles $Z'$-Boson durch unsere Cuts ignoriert wurde.
Ebenfalls nicht ausgeschlossen werden kann die Existenz eines $Z'$-Bosons mit anderen Massen oder anderen bevorzugten Zerfallskanälen, als der untersuchte $Z' \rightarrow t\overline{t}$.