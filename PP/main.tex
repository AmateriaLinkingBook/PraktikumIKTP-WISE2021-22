% Vorlage: https://www.pfsr.de/latex

% -- Anfang Präambel
\documentclass[german,  % Standardmäßig deutsche Eigenarten, englisch -> english
parskip=full,  % Absätze durch Leerzeile trennen
%bibliography=totoc,  % Literatur im Inhaltsverzeichnis (ist unüblich)
%draft,  % TODO: Entwurfsmodus -> entfernen für endgültige Version
]{scrartcl}

\usepackage[utf8]{inputenc}  % Kodierung der Datei
\usepackage[T1]{fontenc}  % Vollen Umfang der Schriftzeichen
\usepackage[ngerman]{babel}  % Sprache auf Deutsch (neue Rechtschreibung)

% Mathematik und Größen
\usepackage{amsmath}
\usepackage[locale=DE,  % deutsche Eigenarten, englisch -> US
separate-uncertainty,  % Unsicherheiten seperat angeben (mit ±)
]{siunitx}
\usepackage{physics}  % Erstellung von Gleichungen vereinfachen

\usepackage{graphicx}  % Bilder einbinden \includegraphics{Pfad/zur/Datei(ohne Dateiendung)}

% Gestaltung
\usepackage{booktabs}  % schönere Tabellen
\usepackage[toc]{multitoc}  % mehrspaltiges Inhaltsverzeichnis
\usepackage{csquotes}  % Anführungszeichen mit \enquote
\usepackage{caption}  % Anpassung der Bildunterschriften, Tabellenüberschriften
\usepackage{subcaption}  % Unterabbildungen, Untertabellen, …
\usepackage{enumitem}  % Listen anpassen
\setlist{itemsep=-10pt}  % Abstände zwischen Listenpunkten verringern

% Manipulation des Seitenstils
\usepackage[headtopline = .5pt]{scrlayer-scrpage}

% Bibliographie
\usepackage[backend=biber]{biblatex}
\addbibresource{bibliography.bib}

% SI-Einheiten darstellen
\usepackage{siunitx}

% Kopf-/Fußzeilen setzen
\pagestyle{scrheadings}  % Stil für die Seite setzen
\clearmainofpairofpagestyles  % Stil zurücksetzen, um ihn neu zu definieren
\automark{section}  % Abschnittsnamen als Seitenbeschriftung verwenden
\ofoot{\pagemark}  % Seitenzahl außen in Fußzeile
\ihead{\headmark}  % Seitenbeschriftung mittig in Kopfzeile

\usepackage[hidelinks]{hyperref}  % Links und weitere PDF-Features

% TODO: Titel und Autor, … festlegen
\newcommand*{\titel}{Auswertung von pp-Kollisionsereignissen}
\newcommand*{\autor}{Sebastian Thiede, Alexander Lettau}
\newcommand*{\abk}{PP}
\newcommand*{\betreuer}{Max Mäerker}
\newcommand*{\messung}{02.12.2021 \& 09.12.2021}
\newcommand*{\ort}{ASB/429}

\hypersetup{pdfauthor={\autor}, pdftitle={\titel}}  % PDF-Metadaten setzen

% automatischen Titel konfigurieren
\titlehead{Praktikum des IKTP \abk \hfill TU Dresden}
\subject{Versuchsprotokoll}
\title{\titel}
\author{\autor}
\date{\begin{tabular}{ll}
Protokoll: & \today\\
Messung: & \messung\\
Ort: & \ort\\
Betreuer: & \betreuer\end{tabular}}

% -- Ende Präambel

\begin{document}
\begin{titlepage}
\maketitle  % Titel setzen
\tableofcontents  % Inhaltsverzeichnis setzen
\end{titlepage}

% ----- DOKUMENT ANFANG -----

\section{Aufgabenstellung}
Die Beschleunigermas­senspektrometrie (AMS) ist ein wichtiges Werkzeug zur Trennung und Detektierung von Nukliden.
Eine der häufigsten Anwendungen ist die Altersbestimmung von Proben aus der Natur durch Messung der Konzentration verschiedener Nuklide, die auf der Erdoberfläche durch kosmische Strahlung entstehen.
Im Gegensatz zu anderen Verfahren der Massenspektrometrie erreicht man bei der AMS eine sehr gute Unterdrückung atomarer und molekularer Isobare.
Die Messzeiten sind relativ kurz und es lassen sich auch kleine Proben untersuchen.

In diesem Versuch wurde der Beschleuniger DREAMS (DREsden AMS) vorgestellt.
Die generelle Handhabung wurde anhand folgender praktischer Aufgaben kennengelernt:
\begin{itemize}
  \item Inbetriebnahme eine Sputter-Ionenquelle und Erzeugung negativer Ionen
  \item Strahlentransport eines Isotopenpaares durch den Beschleuniger
  \item Kalibrierung der Verstärkung der gasgefüllten Ionistationskammer
  \item Aufnahme von Messwerten in der gasgefüllten Ionisationskammer
\end{itemize}

Zur Auswertung sind folgende Aufgaben zu bearbeiten:
\begin{itemize}
  \item Berechnung der Teilchenenergien nach dem Beschleuniger
  \item Plotten des Stromes des Ionenstrahls im Faraday Cup
  \item Abschätzung des Energieverlustes des Ionenstrahls in einer dünnen Folie
  \item Abschätzung des Energieverlustes des Ionenstrahls in Isobutan (Gas in der Ionisationskammer)
  \item Plotten der gemessenen Spektren und Identifikation der Ionen
  \item Berechnung der Konzentration von Radionukliden in einer unbekannten Probe
\end{itemize}


\nocite{*} % alle resourcen auflisten
\printbibliography

% ----- DOKUMENT ENDE -----

\end{document}
